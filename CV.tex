\documentclass[11pt]{article}


\setlength{\parindent}{0pt}
\usepackage{xltxtra}
\usepackage{hyperref}
\hypersetup{hidelinks}
\usepackage{url}
\urlstyle{tt}
\usepackage{xcolor}
\definecolor{CVBlue}{RGB}{23,110,191}
\usepackage{calc}
\usepackage{graphicx}
\usepackage{tikz}
\usetikzlibrary{calc}
\usepackage{fontspec}
\usepackage{xeCJK}
\usepackage{enumitem}
\CJKsetecglue{} %% 取消中文与数字之间的间隙

%% 主文档字体设置
\setmainfont[
    Path = fonts/Main/,
    Extension = .otf,
    BoldFont = texgyretermes-bold.otf, % 加粗字体
]{texgyretermes-regular.otf} % 正文字体

% 中文字体设置
\setCJKmainfont[
    Path = fonts/hansans/,
    Extension = .ttf,
    BoldFont = NotoSansSC-Bold.ttf, % 加粗字体
]{NotoSansSC-Regular.otf} % 正文字体

\usepackage{relsize}
\usepackage{xspace}

% 使用 fontawesome(部分图标)
\usepackage{fontawesome} 

% A4纸,上下左右边距
\usepackage[
    a4paper,
    left=1.2cm,
    right=1.2cm,
    top=1.5cm,
    bottom=1cm,
    nohead
]{geometry}

\renewcommand{\baselinestretch}{1.5} % 行间距设为1.5

\usepackage{titlesec}
\usepackage{enumitem}
\setlist{noitemsep} % 取消列表项间的额外间距
%\setlist{nosep} % 取消所有垂直间距
\setlist[itemize]{topsep=0.25em, leftmargin=*}
\setlist[enumerate]{topsep=0.25em, leftmargin=*}

% --- 用于控制【不同项目之间】的垂直距离 ---
\newlength{\interProjectSpacing}
\setlength{\interProjectSpacing}{0.9em} % <--- 在此调整项目之间的距离
\newcommand{\projectsep}{\vspace{\interProjectSpacing}}

% --- 用于控制【项目标题】与下方【项目描述】的距离 ---
\newlength{\intraProjectTitleSep}
\setlength{\intraProjectTitleSep}{0.4em} % <--- 在此调整标题和描述的距离
\newcommand{\titlebreak}{\\[\intraProjectTitleSep]}

% --- 用于控制【项目描述】与下方【要点列表】的距离 ---
\newlength{\intraProjectListTopSep}
\setlength{\intraProjectListTopSep}{0.2em} % <--- 在此调整描述和列表的距离

% =======================================================================

\titleformat{\section}         % 定制 \section 命令 
{\large\bfseries\raggedright} % 将 section 标题设置为大号、粗体且左对齐
{}{0em}                      % 可用于为所有 section 添加前缀(如“章节...”)
{}                           % 可用于在标题前插入代码
[{\color{CVBlue}\titlerule}]  % 在标题后插入一条横线
\titlespacing*{\section}{0cm}{*1.6}{*1.2}

\begin{document}
\pagenumbering{gobble}

%%%% 利用tikz来定位照片(注释掉,因为无照片)
% \begin{tikzpicture}[remember picture, overlay] 
%     \node[anchor = north east] at ($(current page.north east)+(-2cm,-1.2cm)$) {\includegraphics[height=3cm]{avatar.jpg}};
% \end{tikzpicture}%
%%%% 利用tikz来定位学校Logo(保留并调整大小)
\begin{tikzpicture}[remember picture, overlay] 
    \node[anchor = north west] at ($(current page.north west)+(0.5cm,0.5cm)$) {\includegraphics[height=2.2cm]{zju.png}};
\end{tikzpicture}%

\centerline{\LARGE\bfseries{谢家航}} 

\centerline{\normalsize{\faPhone\ 13615872286 \quad \faEnvelopeO\ \href{mailto:892229840@qq.com}{892229840@qq.com}}} 

% 去除github与个人博客行(原简历未提供)
    
\section{\makebox[\widthof{\faGraduationCap}][c]{\color{CVBlue}\faGraduationCap}\ 教育背景}    
\textbf{浙江大学} \hfill 2022.9 -- 至今\\[0.5em]
\textbf{统计学} 专业 \quad 本科
\begin{itemize}[nosep]
    \item \textbf{GPA:} 3.65/4.3 (82.59/100)
    \item \textbf{主修课程:} 概率论、数理统计、随机过程、数学软件、时间序列分析、多元统计分析等
\end{itemize}

\section{\makebox[\widthof{\faUsers}][c]{\color{CVBlue}\faUsers}\ 项目经历}

% --- 第一个项目:基因测序数据分析 ---
\textbf{基因测序数据分析(Lasso \& 弹性网络)} \hfill 2023.10 -- 2024.01 \titlebreak
在10,000+维基因表达数据中,采用正则化方法筛选疾病相关基因并验证模型稳定性。
\begin{itemize}[nosep, topsep=\intraProjectListTopSep]
    \item 针对高维基因表达数据,采用\textbf{Lasso回归}和弹性网络($\alpha = 0.5$)进行变量选择,筛选出5个与靶标疾病显著相关的基因(FDR $<0.01$)。
    \item 通过\textbf{主成分分析}(PCA)与\textbf{交叉验证}检验模型的稳定性与泛化能力,确保筛选结果的可靠性。
    \item 利用R语言实现数据预处理、建模及可视化,积累了处理生物统计高维数据的实战经验。
\end{itemize}

\projectsep

% --- 第二个项目:金融风控研究 ---
\textbf{金融风控:贝叶斯网络与违约预测优化} \hfill 2024.02 -- 2024.05 \titlebreak
构建贝叶斯网络模型揭示信用变量间的依赖关系,提升违约预测性能。
\begin{itemize}[nosep, topsep=\intraProjectListTopSep]
    \item 构建\textbf{贝叶斯网络}模型捕捉风险变量间的概率依赖结构,结合\textbf{MCMC}算法进行参数后验估计,增强模型可解释性。
    \item 通过引入概率图模型,将违约预测的\textbf{F1值从0.78提升至0.87},显著改善了风控模型的精准率与召回率。
    \item 使用Python进行模型实现与评估,包括网络结构学习、参数估计及性能对比。
\end{itemize}

\section{\makebox[\widthof{\faCogs}][c]{\color{CVBlue}\faCogs}\ 技术栈}
\begin{itemize}[nosep]
    \item \textbf{编程语言:} Python, R, (基础: C++/Java)
    \item \textbf{统计与建模:} 概率论与数理统计,多元统计分析,时间序列分析,贝叶斯网络,Lasso回归,PCA
    \item \textbf{工具与平台:} Git, RStudio, Jupyter, LaTeX, Linux
\end{itemize}

% 原简历无获奖经历,故省略该section
% \section{\makebox[\widthof{\faGraduationCap}][c]{\color{CVBlue}\faList}\ 获奖情况}
% \begin{itemize}
%     \item 待补充
% \end{itemize}
    
\section{\makebox[\widthof{\faInfo}][c]{\color{CVBlue}\faInfo}\ 其他信息}
\begin{itemize}[parsep=0.5ex]
    \item \textbf{英语水平:} 英语六级(CET-6),具备良好的书面与口语沟通能力。
    \item \textbf{自我评价:} 具备扎实的统计专业基础和较强的自主学习能力;有良好的抗压能力和团队协作精神;善于沟通,学术表达清晰;对数据科学在实际问题中的应用充满热情。
\end{itemize}

\end{document}
